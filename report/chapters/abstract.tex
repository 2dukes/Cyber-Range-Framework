\chapter*{Resumo}
%\addcontentsline{toc}{chapter}{Resumo}

A infraestrutura de uma organização pressupõe que os profissionais de cibersegurança têm conhecimento específico em administrar e proteger a mesma contra ameaças externas. Sem este nível de especialização, informação sensível poderia ser comprometida e exposta a atores maliciosos, cujo intuito final seria causar dano aos sistemas chave da infraestrutura. Estes tipo de ataques tende a ficar cada vez mais especializado fazendo com que os profissionais na área de cibersegurança tenham de garantir proficiência na área onde operam. Naturalmente, recomendações incluem a criação de cenários avançados para treino prático onde situações realistas são consideradas com o intuito de ajudar os profissionais a ganhar conhecimento detalhado. No entanto, problemas relacionados com o custo elevado da infraestrutura e dificuldades no processo de \textit{deployment}, sobretudo devido ao processo manual de pré-configuração do \textit{software} necessário para o treino e da dependência em VMs estáticas, estão associadas a uma elevada quantidade de trabalho.

Para facilitar este processo, o nosso trabalho aborda o uso de \textit{Infrastructure as Code} (\textit{IAC}) e \textit{DevOps} como formar de automatizar o processo de \textit{deployment} de \textit{cyber ranges}. Uma abordagem estreitamente relacionada com virtualização e \textit{containerization}, no que diz respeito à estrutura subjacente do código da infraestrutura, permite aliviar todo este processo. Notavelmente, o uso de programas relacionados com \textit{IaC}, como o Ansible, facilitam todo o processo de configuração e provisionamento de uma rede. Desta forma, o foco inicial do trabalho começa por entender quais as necessidades face ao presente Estado da Arte e demonstrar os benefícios inerentes à nova metodologia de trabalho adotada. Por último, variadas vulnerabilidades serão exploradas, a maioria relacionada com \textit{Privilege Escalation}, \textit{Remote Code Execution} e \textit{Incident Forensics}, permitindo assim a aprimoração de \textit{skills} relacionadas com cenários \textit{Red team} e \textit{Blue team}. A análise deste tipo de ataques e a exploração destas vulnerabilidades serão realizadas com a devida segurança devido à existência de \textit{sandboxing} proveniente dos \textit{containers}.

Os resultados esperados oscilam em torno do uso de \textit{IaC} para criar um grupo de \textit{cyber ranges} projetadas com um conjunto de desafios específicos. O principal objetivo é começar por cenários mais simples, aumentando a complexidade dos mesmos até que redes de nível empresarial sejam atingidas. Esta lógica implica desenvolver um conjunto de \textit{playbooks} com o intuito de serem executados numa máquina ou laboratório, assegurando um estado consistente da rede final. Espera-se que esta execução seja rápida e de baixo custo, permitindo ao profissional aprofundar o seu conhecimento tendo por base um conjunto variado de situações.

Recentemente, soluções baseadas em \textit{DevOps} tendem a ser as mais eficazes face aos problemas derivados de abordagens mais antigas, no que diz respeito à configuração deste tipo de cenários. Em suma, pretende-se com este trabalho melhor preparar os especialistas na área de cibersegurança, assegurando os princípios do \textit{National Institute of Standards and Technology (NIST) Cybersecurity Framework}, nomeadamente: prevenir, detetar, mitigar e recuperar. \\

\newcommand{\titles}[2]{\noindent\textbf{#1:} #2\\[2mm]}

\titles{Palavras-chave}{\textit{Infrastructure as Code, DevOps, Cyber Range, Cibersegurança, Virtualização}}

\chapter*{Abstract}
%\addcontentsline{toc}{chapter}{Abstract}

An organization's infrastructure rests upon the premise that cybersecurity professionals have specific knowledge in administrating and protecting it against outside threats. Without this expertise, sensitive information could be leaked to malicious actors and cause damage to critical systems. These attacks tend to become increasingly specialized, meaning cybersecurity professionals must ensure proficiency in specific areas. Naturally, recommendations include creating advanced practical training scenarios considering realistic situations to help trainees gain detailed knowledge. However, the caveats of high-cost infrastructure and difficulties in the deployment process of this kind of system, primarily due to the manual process of pre-configuring software needed for the training and relying on a set of static Virtual Machines, may take much work to circumvent.

In order to facilitate this process, our work addresses the use of Infrastructure as Code (IaC) and DevOps to automate the deployment of cyber ranges. An approach closely related to virtualization and containerization as the code's underlying infrastructure helps lay down this burden. Notably, placing emphasis on using IaC tools like Ansible eases the process of configuration management and provisioning of a network. Therefore, we start by focusing on understanding what the \textit{state-of-the-art} perspectives lack and showcasing the benefits of this new working outlook. Lastly, we explore several vulnerabilities mostly related to Privilege Escalation, Remote Code Execution attacks, and Incident Forensics, allowing for improved skills concerning Red team and Blue team scenarios. The analysis of the attacks and exploitation of such vulnerabilities are carried out safely due to a sandbox environment.

The expected results revolve around using IaC to deploy a set of purposely-designed cyber ranges with specific challenges. The main goal is to start from simple testbeds and increase complexity until we reach realistic enterprise-level networks. Thus, this entails having a set of playbooks that can be run in a machine or laboratory, assuring the final state of the network is consistent. We expect this deployment strategy to be quick and cost-effective, allowing the trainee to get deep insight into a wide range of situations.

Nowadays, DevOps solutions work as a silver bullet against the issues derived from old-case-driven approaches for setting up scenarios. In short, the contribution to better prepare specialists in ensuring that the principles of the National Institute of Standards and Technology (NIST) Cybersecurity Framework hold, namely: prevent, detect, mitigate and recover, is one of the key takeaways of this work.\\

\titles{Keywords}{\textit{Infrastructure as Code, DevOps, Cyber Range, Cybersecurity, Virtualization}}
