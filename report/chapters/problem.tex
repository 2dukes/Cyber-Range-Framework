\chapter{Problem Statement}\label{chap:problem_statement}

\minitoc

\section{Problem under Study} \label{sec:research_problem}

Cyber ranges play a crucial role in cybersecurity training and are widely recognized as valuable tools for developing and enhancing skills in the field. The benefits of cyber ranges include providing realistic simulations in controlled and secure environments that replicate real-world networks and attack scenarios, practical skills development related to critical thinking and problem-solving, and experiential learning because trainees experiment with different tools, techniques, and strategies, test their effectiveness, and learn both from successes and failures. All these activities are performed in a risk-free environment, which does not pose any chance of causing damage to real systems or networks and helps build a deeper understanding of the cybersecurity landscape, fostering the ability to adapt and respond to evolving threats. 

Several platforms can be used to improve skills in the cybersecurity field. We previously mentioned Hack The Box and TryHackMe, but many more businesses provide these services. Cybersecurity training is an essential subject that companies, agencies, and governments care about because knowledge gaps in such areas may endanger many citizens and organizations. The European Union, for instance, created an initiative called \textit{CyberSec4Europe}\footnote{\url{https://cybersec4europe.eu/}} which aims to strengthen cybersecurity research and innovation, encourage collaboration and knowledge sharing, address skill gaps, and enhance overall preparedness in the field. 

From the literature review presented in Chapter \ref{chap:sota} (p. \pageref{chap:sota}) old-case approaches involve the use of high-cost infrastructure relying too much on Virtual Machines and physical hardware. Instead, container-based solutions are starting to emerge but still rely heavily on custom cyber range description files. Many of the studied cyber range frameworks pose scalability and expansibility issues related to the lack of modularity in the chosen design or the non-inclusion of Infrastructure as Code tools. The lack of community support and the small amount of open-sourced frameworks also seems to be an issue, which may be evidence that some projects were suspended. Furthermore, many approached frameworks do not reference up-to-date vulnerabilities and attacks. Lastly, regarding how realistic a cyber range can be, some frameworks did not address this because they did not have a general network structure common to each scenario that could simulate an enterprise network. 

To address the issues presented in the literature review, the following research questions can define the problem under study:

\begin{enumerate}
    \item How to overcome the weaknesses present in the current \textit{state-of-the-art} on cyber ranges, namely high-cost deployment of infrastructure, custom description files that make scenario extensions harder, and lack of randomization? 
    \item How to develop lightweight scenarios with the complexity of an enterprise-level network without using old-case-driven approaches based on physical hardware and Virtual Machines?
    \item How to automatically deploy, provision, and configure cyber ranges in a scalable way?
    \item How to simplify the creation process of scenarios for the end-user?
\end{enumerate}

Having in mind these questions, the following hypothesis was considered:\\

\leftskip=1.25cm\rightskip=1.25cm

\textbf{H:} \textit{``Using an approach heavily relying upon DevOps, Infrastructure as Code and containerization, it is possible to automatically deploy and provision, in a cost-effective manner, a set of vulnerable enterprise-level scenarios, ensuring practical cybersecurity training.''}

\leftskip=0cm\rightskip=0cm

\section{Hypothesis Validation} \label{sec:hypothesis_validation}

To answer the hypothesis formed in Section \ref{sec:research_problem} (p. \pageref{sec:research_problem}), this work intends to develop a cyber range framework that builds upon the weaknesses of the presented projects in the \textit{state-of-the-art}, providing a unique solution based on IaC and containers. Moreover, we intend to allow local and cloud deployments running on a local computer or in a cloud instance. The latter provides greater security due to the security restrictions the cloud provider applies to isolate each virtual instance.  

To achieve such a framework, we use IaC tools like Ansible to automate configuration and provisioning of the cyber range while thinking of future scenarios' expansion, which should be more straightforward, as commonly known technologies were used. Using Docker containers, we can develop lightweight complex networks with a wide variety of network services. Using IaC, we avoid using a custom approach to each cyber range deployment. We must consider the dilemma of using Virtual Machines: on the one hand, they cause significant overhead in terms of resource consumption; on the other hand, VMs allow us to explore a broader range of vulnerabilities, for instance, kernel-related ones. Still, we use them in the context of Windows-based scenarios, instead of just sticking with Linux-based scenarios, given our goal of simulating a wide range of scenario types that follow the concerns of today's cyber attacks. Then, we intend to address randomization by combining attack graphs consisting of different kinds of attacks on each scenario, as well as randomizing some network configurations, challenging the trainee by introducing different possibilities to a higher level of privilege. Lastly, we intend to create a mesh network with the machines relevant for the trainee to successfully solve the challenge when thinking of remote deployments.

To conclude this topic, cyber ranges are essential in the context of education and cybersecurity professionals to significantly improve their skills. Trainees see themselves as forced to perform experiments with real-world environment simulations that will better prepare them for security incident situations. Considering all this, we argue that the above-presented hypothesis is \textbf{plausible}.

\section{Summary} \label{sec:problem_summary}

This chapter has drawn a line on where the \textit{state-of-the-art} research is and some of its issues. We recognize that building cyber ranges capable of validating our hypothesis poses several challenges that will be addressed in Chapter \ref{chap:validation} (p. \pageref{chap:validation}), which presents the actual work behind the hypothesis validation and how the project development took place. With this, the structure followed by this chapter started with the exploration of the problem under study, introduced in Section \ref{sec:research_problem} (p. \pageref{sec:research_problem}), along with the analysis of the literature review placed in Chapter \ref{chap:sota} (p. \pageref{chap:sota}). Section \ref{sec:hypothesis_validation} (p. \pageref{sec:hypothesis_validation}) presents the scope and validation of the hypothesis, as well as the definition of the driving concept behind its validation.